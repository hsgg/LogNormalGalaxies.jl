%\documentclass[twocolumn]{aastex62}
\documentclass[aps,prd,reprint,floatfix,superscriptaddress,showkeys,nofootinbib]{revtex4-1}

%\bibliographystyle{apsrev}  % doesn't set links, no title
\bibliographystyle{apsrev4-2-truncate}
\setcitestyle{authoryear,round}
\setlength{\bibsep}{4pt}



\usepackage[utf8]{inputenc}
\usepackage[T1]{fontenc}

%\received{\today}
%\revised{\today}
%\accepted{\today}
%\submitjournal{ApJ}
%\shorttitle{Effect of FoG in $C_\ell$ Space}
%\shortauthors{Grasshorn Gebhardt et al.}


\usepackage{graphicx}
\usepackage{amsmath}
\usepackage{amssymb}
\usepackage{verbatim}
\usepackage{rotate}
\usepackage{color}
%\usepackage{arydshln}
\usepackage{bm}
\usepackage{aas_macros}
\usepackage{tikz}
\usetikzlibrary{calc}

% https://tex.stackexchange.com/questions/192610/use-emulateapj-aastex-with-siunitx#269074
\usepackage{savesym}
\savesymbol{tablenum}
\usepackage{siunitx}
\sisetup{group-separator={,},group-digits=integer}
\restoresymbol{SIX}{tablenum}


\DeclareFontFamily{OT1}{pzc}{}
\DeclareFontShape{OT1}{pzc}{m}{it}%
            {<-> s * [1.10] pzcmi7t}{}
\DeclareMathAlphabet{\mathscr}{OT1}{pzc}%
                                {m}{it}


\definecolor{RedWine}{rgb}{0.743,0,0}
\definecolor{GrassGreen}{rgb}{0.125,0.75,0.125}
\definecolor{RoyalBlue}{rgb}{0.25,0.41,0.88}
\renewcommand{\dj}[1]{\textcolor{RedWine}{\textbf{[DJ: #1]}}}
\newcommand{\hg}[1]{\textcolor{GrassGreen}{\textbf{(HG: #1)}}}


%% definition for equations
\newcommand{\be}{\begin{equation}}
\newcommand{\ee}{\end{equation}}
\newcommand{\bea}{\begin{eqnarray}}
\newcommand{\eea}{\end{eqnarray}}
\def\ba#1\ea{\begin{align}#1\end{align}}

\def\({\left(}
\def\){\right)}
\def\<{\left\langle}
\def\>{\right\rangle}


%\newcommand{\refeq}[1]{Eq.~(\ref{eq:#1})}
%\newcommand{\refeqs}[2]{Eqs.~(\ref{eq:#1})--(\ref{eq:#2})}
%\newcommand{\reffig}[1]{Fig.~\ref{fig:#1}}
%\newcommand{\refsec}[1]{Sec.~\ref{sec:#1}}
%\newcommand{\refapp}[1]{App.~\ref{app:#1}}
%\newcommand{\reftab}[1]{Tab.~\ref{tab:#1}}
\newcommand{\refeq}[1]{\cref{eq:#1}}
\newcommand{\refeqs}[2]{\crefrange{eq:#1}{eq:#2}}
\newcommand{\reffig}[1]{\cref{fig:#1}}
\newcommand{\refsec}[1]{\cref{sec:#1}}
\newcommand{\refapp}[1]{\cref{app:#1}}
\newcommand{\reftab}[1]{\cref{tab:#1}}

\newcommand{\vs}{\nonumber\\} 

\def\vK{{\bm{K}}}
\def\vR{{\bm{R}}}
\def\vr{{\bm{r}}}
\def\vx{{\bm{x}}}
\def\vv{{\bm{v}}}
\def\vp{{\bm{p}}}
\def\vvs{{\bm{s}}}
\def\vk{{\bm{k}}}
\def\vY{{\bm{Y}}}
\def\vy{{\bm{y}}}
\def\vq{{\bm{q}}}
\def\vu{{\bm{u}}}
\def\vgam{{\bm{\gamma}}}
\def\vPsi{{\bm{\Psi}}}

\def\vecl{\vec{l}}
\def\vecL{\vec{L}}
\def\vecv{\vec{v}}
\def\veck{\vec{k}}
\def\vecp{\vec{p}}
\def\vecx{\vec{x}}
\def\vect{\vec{\theta}}

\def\JM{{(\ell m)}}
\def\lm{{(\ell m)}}
\def\zt{\tilde{z}}
\def\chit{{\tilde\chi}}
\def\bt{b_e}
\def\H{\mathcal{H}}
\def\Gaunt#1#2#3#4#5#6{\mathcal{G}^{#1}_{#4}{}^{#2}_{#5}{}^{#3}_{#6}}
\def\PA{\mathrm{PA}}

\def\ehat{{\hat{\bm{e}}}}
\def\nhat{{\hat{\bm{n}}}}
\def\khat{{\hat{\bm{k}}}}
\def\Khat{{\hat{\bm{K}}}}
\def\xhat{{\hat{\bm{x}}}}
\def\vhat{{\hat{\bm{v}}}}
\def\phat{{\hat{\bm{p}}}}
\def\qhat{{\hat{\bm{q}}}}
\def\rhat{{\hat{\bm{r}}}}
\def\Rhat{{\hat{\bm{R}}}}
\def\shat{{\hat{\bm{s}}}}
\def\yhat{{\hat{\bm{y}}}}
\def\zhat{{\hat{\bm{z}}}}
\def\Nbar{{\bar{N}}}
\def\nbar{{\bar{n}}}

\def\fnl{f_\mathrm{NL}}
\def\erfc{\mathrm{erfc}}
\def\Pp{P_\Phi}
\DeclareMathOperator{\erf}{erf}

\def\orderof{\mathcal{O}}
\newcommand{\half}[1][1]{\ensuremath{\frac{#1}{2}}}
\newcommand{\quarter}[1][1]{\ensuremath{\frac{#1}{4}}}

\DeclareSIUnit \parsec {pc}
\DeclareSIUnit \h {\text{$h$}}
\DeclareSIUnit \year {yr}
\DeclareSIUnit \solarmass {M_\odot}
\DeclareSIUnit \Mpc {\mega\parsec}
\newcommand{\hMpc}{\ensuremath{\mega\parsec\per\h}}

\def\tot{\rm total}
\def\oii{\mathrm{OII}}
\def\oiii{\mathrm{OIII}}
\def\lae{\mathrm{LAE}}
\def\lin{\mathrm{lin}}
\def\hae{\mathrm{HAE}}

\def\th{\mathrm{th}}
\def\obs{\mathrm{obs}}
\def\proj{\mathrm{proj}}
\def\true{\mathrm{true}}
\def\inferred{\mathrm{inferred}}
\def\est{\mathrm{est}}  % current estimate (in an iterative setting)
\def\min{\mathrm{min}}
\def\max{\mathrm{max}}
\def\mid{\mathrm{mid}}
\def\sky{\mathrm{sky}}
\def\fill{\mathrm{fill}}

\def\xlae{x_\lae}
\def\xoii{x_\oii}
\def\Plin{P_{\rm Lin}}

\def\tr{\mathrm{tr}}
\def\atanh{\mathrm{atanh}}

\def\Like{\mathfrak{L}}
\def\f{\mathfrak{f}}
\def\g{\mathfrak{g}}
\def\ffid{\f_\mathrm{true}}
\def\gfid{\g_\mathrm{true}}
\def\Q{{\cal Q}}

\def\dd{\mathrm{d}}
\def\deriv#1{\mathrm{\romannumeral #1}}

% https://tex.stackexchange.com/questions/33538/how-to-get-an-approximately-proportional-to-symbol
\def\myapp#1#2{%
  \mathrel{%
    \setbox0=\hbox{$#1\sim$}%
    \setbox2=\hbox{%
      \rlap{\hbox{$#1\propto$}}%
      \lower1.1\ht0\box0%
    }%
    \raise0.25\ht2\box2%
  }%
}
\def\approxpropto{\mathpalette\myapp\relax}

\newcommand{\incgraph}[2][0.49]{\includegraphics[width=#1\textwidth]{#2}}

\newcommand{\changed}[1]{\textcolor{red}{#1}}


\usepackage[colorlinks]{hyperref}
\usepackage[capitalise]{cleveref}
\newcommand{\crefrangeconjunction}{--}




\begin{document}

\title{Log-normal simulations}

%\author[0000-0002-8158-0523]{Henry S. Grasshorn Gebhardt}
\author{Henry S. \surname{Grasshorn Gebhardt}}
\email{hsg113@psu.edu}
%\affiliation{Department of Astronomy and Astrophysics and
%    Institute for Gravitation and the Cosmos, \\
%    The Pennsylvania State University, University Park, PA 16802, USA
%}


\begin{abstract}
  Here we give some details on log-normal simulations. The Appendix gives a
  power spectrum estimator with $\mu$-leakage correction.
\end{abstract}

\keywords{cosmology; large-scale structure}


\maketitle
\tableofcontents


%%%%%%%%%%%%%%%%%%%%%%%%%%%%%%%%%%%%%%%%%%%%%%%%%%%%%%%%%%%%%%%%%%%%%%%%%%%%%%
\section{Introduction}
The density contrast field $\delta(\vx)$ is often treated as Gaussian. However,
this can lead to $\delta<-1$, which is forbidden by construction. A more
realistic model, therefore, is a log-normal distribution for $\delta$, which is
naturally constrained to $\delta\geq-1$.

In this note we investigate how to derive a log-normal simulation. We
mostly follow \citet{Agrawal+:2017JCAP...10..003A}. This was first done by
\citet{Coles+:1991MNRAS.248....1C}.

Other codes doing log-normal simulations are FLASK
\citep{Xavier+:2016MNRAS.459.3693X} and NBodyKit
\citep{Hand+:2017JCAP...07..002H}. FLASK seems to be overly complicated. You
provide it $C_\ell(r,r')$ with all the tomographic cross-correlations to make a
simulation.

Log-normal is not fully described by all its moments
\citep{Coles+:1991MNRAS.248....1C}. That is, there is another distinct
distribution that has the same moments.

\citet{Bond+:2000ApJ...533...19B} describe the effect of non-Gaussian
distribution on the power spectrum estimation. They provide approximations for
the distribution of the $C_\ell$, and a nice iterative procedure for estimating
cosmological parameters from the estimated $C_\ell$.
\citet{Efstathiou+:2001MNRAS.325.1603E} apply this to the APM survey.





%%%%%%%%%%%%%%%%%%%%%%%%%
\section{Log-normal density field}
%%%%%%%%%%%%%%%
\begin{figure}
  \incgraph{../figs/pk_pkG}
  \caption{Power spectrum for $\delta$ and $G$.}
  \label{fig:pk_pkG}
\end{figure}
%%%%%%%%%%%%%%%
We define the log-transformed density field
\ba
G(\vx) &= \ln(1+\delta(\vx)) - \<\ln(1+\delta(\vx))\>_\vx\,.
\ea
Then,
\ba
\label{eq:delta_deltaG}
\delta(\vx) &= e^{-\frac12\sigma_G^2 + G(\vx)} - 1\,,
\ea
where $\sigma_G^2\equiv\<G(\vx)^2\>_\vx$ follows from the requirement that the
average density contrast $\<\delta\>_\vx=0$ must vanish, and we used that
$G(\vx)$ is a Gaussian field with vanishing mean so that
\ba
\<e^{G}\>
&= \sum_n \frac{\<G^n\>}{n!}
= \sum_n \frac{\<G^{2n}\>}{(2n)!}
= \sum_n \frac{\#\<G^2\>^n}{(2n)!}
\\
&= \sum_n \frac{\(\frac12\<G^2\>\)^n}{n!}
= e^{\frac12\<G^2\>}\,,
\ea
where Wick's theorem was used.
Then, defining
\ba
\xi(\vr) &= \<\delta(\vx)\delta(\vx+\vr)\>_\vx\,,
\ea
we get
\ba
\label{eq:xiG}
\xi_G(\vr) &= \ln(1 + \xi(\vr))\,,
\ea
where we assumed a homogeneous universe.
To get the normally-distributed power spectrum $P_G(\vq)$, we first
Fourier-transform the power spectrum $P(\vk)$, apply \refeq{xiG}, and then
transform to $P_G(\vk)$. That is,
\ba
P_G(k)
&= \int\dd^3\vr \, e^{-i\vk\cdot\vr}
\, \xi_G(\vr)
\vs&
= (2\pi)^3\int_0^\infty\frac{r^2\dd r}{2\pi^2}
\, j_0(kr)
\, \xi_G(r)
\\
\xi(r)
&= \int\frac{\dd^3q}{(2\pi)^3} \, e^{i\vq\cdot\vr} \, P(\vq)
\vs&
= \int_0^\infty\frac{q^2\dd q}{2\pi^2}
\, j_0(qr)
\, P(q)
\ea
This allows us to draw the Gaussian field $\delta_G(\vk)$, which we transform
to configuration space, then apply \refeq{delta_deltaG} to obtain the density
field.

\cref{fig:pk_pkG} shows a linear matter power spectrum with different
amplitudes, comparing the power spectrum for the lognormal field $\delta$ and
the normal field $G$.


%%%%%%%%%%%%%%%%%%%%%%%%%%%%%%%%%%%%%%%%%%%%%%%%%%%%%
\subsection{Predicting sigma G squared}
\ba
\sigma_G^2
&= \<G^2(\vx)\>
= \xi^G(0)
= \ln(1 + \xi(0))
\vs&
= \ln\(1 + \int\frac{k^2\dd k}{2\pi^2}\,P(k)\).
\ea
This would be the estimate for the fluctuations if the resolution of our
simulation was infinite. To get the relevant $\sigma_G^2$ at finite resolution,
we include the pixel window,
\ba
\sigma_G^2
&= \int\dd^3r\,W(\vr)\int\dd^3x\,W(\vx)\<G(\vr)G(\vx)\>
\\
&= \int\dd^3r\,W(\vr)\int\dd^3x\,W(\vx)\,\xi_G(\vr-\vx)
\\
&= \int\frac{\dd^3k}{(2\pi)^3}\,|\widetilde W(\vk)|^2\,P_G(k)\,.
\ea
Numerically, I find that $\sigma_G^2$ from one simulation is about 24\% higher
for the matter field, and 20\% for the galaxy field. This may due to the
calculation assuming spherical cells, while our cells are cubic.



%%%%%%%%%%%%%%%%%%%%%%%%%%%%%%%%%%%%%%%%%%%%%%%%%%%%%
\subsection{The Velocity field}
The velocity field to first order is given by
\ba
\vv(\vk) &= aH\vu\,,
\ea
where
\ba
\vu(\vk) &= if\,\frac{\vk}{k^2}\,\delta_m(\vk)\,.
\ea
We assume that galaxies have the same velocity as the matter.
RSD puts galaxies at
\ba
\vvs &= \vr + \frac{\vv\cdot\rhat}{aH}\,\rhat
= \vr + \(\vu\cdot\rhat\)\rhat\,.
\ea
Hence, calculating $\vu$ is needed to get velocities.


%%%%%%%%%%%%%%%%%%%%%%%%%%%%%%%%%%%%%%%%%%%%%%%%%%%%%
\subsection{Drawing galaxies}
We draw galaxies from the real-space density contrast. The number of
galaxies in each grid cell is drawn from a Poisson distribution, and the galaxy
positions are uniformly drawn inside the grid cell. We assign the same velocity
to each galaxy within a grid cell. This is similar to NGP assignment.

As a second option, we also allow drawing the position twice: First, uniformly
within the grid cell, then uniformly within a cubic cloud around the current
position. This is a convolution of two top-hat distributions. Combined with the
probability of the neighboring cells, this is the same as drawing from a
tri-linear interpolation distribution (since Poisson probabilities add), and it
is similar to CiC on the estimator side.

Velocities can be assigned by several options: Using the nearest grid point, or
by linearly interpolating neighbors.
To simulate a FoG, we add Gaussian noise to the velocities.





%%%%%%%%%%%%%%%%%%%%%%%%%%%%%%%%%%%%%%%%%%%%%%%%%%%%%%%%%%%%%%%%%%%%%%%%%%%%%%
\section{Conclusion}
Log-normals for the win!


%%%%%%%%%%%%%%%%%%%%%%%%%%%%%%%%%%%%%%%%%%%%%%%%%%%%%%%%%%%%%%%%%%%%%%%%%%%%%%
%\acknowledgements
%The authors thank God for pretending to point at us.


%%%%%%%%%%%%%%%%%%%%%%%%%%%%%%%%%%%%%%%%%%%%%%%%%%%%%%%%%%%%%%%%%%%%%%%%%%%%%%
\bibliography{../references}
%%%%%%%%%%%%%%%%%%%%%%%%%%%%%%%%%%%%%%%%%%%%%%%%%%%%%%%%%%%%%%%%%%%%%%%%%%%%%%


%%%%%%%%%%%%%%%%%%%%%%%%%%%%%%%%%%%%%%%%%%%%%%%%%%%%%%%%%%%%%%%%%%%%%%%%%%%%%%
\appendix
%%%%%%%%%%%%%%%%%%%%%%%%%%%%%%%%%%%%%%%%%%%%%%%%%%%%%%%%%%%%%%%%%%%%%%%%%%%%%%


%%%%%%%%%%%%%%%%%%%%%%%%%%%%%%%%%%%%%%%%%%%%%%%%%%%%%%%%%%%%%%%%%%%%%%%%%%%%%%
\section{Log-normal probability distribution}
We assume that $\delta(\vr)$ is log-normal distributed. The average must
vanish, by definition, and the power spectrum measures its variance.
The log-normal is given by the probability distribution
\ba
p(\delta) &= \frac{1}{\(1+\delta\)\sqrt{2\pi\sigma^2}}\,
e^{-\frac{\(\ln(1+\delta) - \mu\)^2}{2\sigma^2}}\,.
\ea
The mean and variance are
\ba
\<\delta\> &= e^{\mu + \frac12\sigma^2} - 1 \,, \\
%\<(1+\delta)^2\> - \<1+\delta\>^2 &= e^{2\mu + \sigma^2} \(e^{\sigma^2} - 1\) \,.
%\\
\<\delta^2\> -\<\delta\>^2 &= e^{2\mu + \sigma^2} \(e^{\sigma^2} - 1\) \,.
\ea
Requiring $\<\delta\>=0$, we get
\ba
\<\delta^2\> &= e^{\sigma^2} - 1\,.
\ea
To get the correlation function, we need to get the covariance between
$\delta$'s at different locations in space. The log-normal distribution for
this is given in \citet{Coles+:1991MNRAS.248....1C}:
\ba
p(\vec\delta)
             &= \frac{1}{\sqrt{(2\pi)^N\det{\Sigma_G}}}
             \left[ \prod_s \frac{1}{1 + \delta_s} \right]
\vs&\quad\times
             e^{-\frac12\sum_{ij}[\ln(1+\delta_i)-\mu_i] \(\Sigma_G^{-1}\)_{ij} [\ln(1+\delta_j) - \mu_j]}
\ea
for the density contrast $\delta_i=\delta(\vr_i)$. Define
\ba
G_i &= \ln(1+\delta_i) - \mu_i \,, \\
p_G(\vec G)
    &= \frac{1}{\sqrt{(2\pi)^N\det{\Sigma_G}}}
    \, e^{-\frac12\sum_{ij}G_i \(\Sigma_G^{-1}\)_{ij} G_j}\,.
\ea
Then, the mean and variance are
\ba
\<\delta_i\>
    &= e^{\mu_i + \frac12\Sigma_{ii}} - 1 \,,\\
\<\(\delta_i - \<\delta_i\>\)\(\delta_j - \<\delta_j\>\)\>
    &= e^{\mu_i + \mu_j + \frac12\(\Sigma_{ii} + \Sigma_{jj}\)}
    \(e^{\Sigma_{ij}} - 1\)\,.
\ea
Requiring the average density contrast to vanish, we get
\ba
\mu_i &= -\frac12\Sigma_{ii}\,,
\\
\<\delta_i \delta_j\>
    &= e^{\Sigma_{ij}} - 1\,.
\ea



%%%%%%%%%%%%%%%%%%%%%%%%%%%%%%%%%%%%%%%%%%%%%%%%%%%%%
\section{Power Spectrum Multipoles with FoG}
Here we aim to incorporate the FoG effect in the multipoles for the plane
parallel limit. Also, see \citet{Lee:2018JCAP...02..039L}.
Our model for the power spectrum is
\ba
P(k,\mu) &= b^2 (1+\beta\mu^2)^2 \, e^{-\sigma_u^2k^2\mu^2}\,P(k)\,,
\ea
where $\beta=f/b$, and to first order we expect
\ba
\sigma_u^2 &= \frac{f}{3}\int\frac{\dd^3k}{(2\pi)^3}\,\frac{P_m(k)}{k^2}\,.
\ea
The multipole power spectrum is, then,
\ba
P_\ell(k) &= \int_{-1}^1\frac{\dd\mu}{2}\,\mathcal{L}_\ell(\mu)\,P(k,\mu) \\
          &= b^2\big[I^\ell_0(x)+2\beta I^\ell_2(x)+\beta^2I^\ell_4(x)\big] P(k)\,,
\ea
where we defined $x = \sigma_u k$ and
\ba
I^\ell_n(x) &= \int_{-1}^1 \frac{\dd\mu}{2} \, \mathcal{L}_\ell(\mu) \, \mu^n \, e^{-x^2\mu^2}\,.
\ea



%%%%%%%%%%%%%%%%%%%%%%%%%%%%%%%%%%%%%%%%%%%%%%%%%%%%%%%%%%%%%%%%%%%%%%%%%%%%%%
\section{NBodyKit}
(This section very incomplete!)
NBodyKit does the following in its LogNormalCatalog class. Note, that this
account is likely incomplete. First, generate white noise density contrast and
displacement fields
\ba
\delta_G(\vk) &= \sqrt{\frac{P(k)}{2}}\left[n_1(\vk) + i\,n_2(\vk)\right],\\
\vPsi(\vk) &= i\,\frac{\vk}{k^2}\,\delta_G(\vk)\,,
\ea
where $n_i(\vk)$ are sampled from a normal distribution with unit variance.
Convert to real space,
\ba
\delta_G(\vx) &= \int\frac{\dd^3 k}{(2\pi)^3}\,e^{i\vk\cdot\vx}\,\delta_G(\vk)
\,,\\
\vPsi(\vx) &= \int\frac{\dd^3 k}{(2\pi)^3}\,e^{i\vk\cdot\vx}\,\vPsi(\vk)\,.
\ea
Next, we ensure that the density contrast is within
$-1\leq\delta(\vx)<\infty$. We start with the Lagrangian bias
\ba
b_L &= b - 1\,,
\ea
and we perform a lognormal transform
\ba
1+\delta(\vq) &= \frac{1}{A}\,e^{b_L \delta_G(\vx)}\,,
\ea
where $A$ is chosen such that
\ba
\<1+\delta(\vq)\> &= 1\,.
\ea
We can derive it to get
\ba
A
&= \<e^{b_L \delta_G(\vq)}\>
= \sum_m \frac{\<[b_L \delta_G(\vq)]^m\>}{m!}
\vs
&= \sum_n \frac{(2n-1)!!\<[b_L\delta_G(\vq)]^2\>^n}{(2n)!}
\vs
&= \sum_n \frac{(2n-1)(2n-3)\cdots 1\<[b_L\delta_G(\vq)]^2\>^n}{(2n)(2n-1)(2n-2)\cdots1}
\vs
&= \sum_n \frac{\<[b_L\delta_G(\vq)]^2\>^n}{(2n)(2n-2)\cdots1}
= \sum_n \frac{\<[b_L\delta_G(\vq)]^2\>^n}{2^n(n)(n-1)\cdots1}
\vs
&= \sum_n \frac{\<\frac12[b_L\delta_G(\vq)]^2\>^n}{n!}
= \exp\!\(\frac12\,b_L^2\<\delta_G^2(\vq)\>\).
\ea
Then we draw Poisson points at each Lagrangian position $\vq$ with mean
$[1+\delta(\vq)]\,\nbar$. That is, the number of galaxies within a cell
centered at $\vq$ is
\ba
N(\vq) &\sim \mathrm{Poisson}\([1+\delta(\vq)]\,\nbar\).
\ea
Within each cell the galaxies are shifted by a uniform random shift. That gives
the number of particles at a given Lagrangian position $\vq_i$ with
displacement $\vPsi_i$. Finally, the particles are moved with the Zel'dovich
displacement
\ba
\vx_i &= \vq_i + \vPsi_i\,,
\ea
and the velocity is
\ba
\vv_i &= faH \vPsi_i\,.
\ea

There are some differences. First, the displacement field is added to the
Lagrangian position. Second, no Hankel transform of the power spectrum. The
Hankel transfrom is there to ensure the power spectrum of the density field
equals the input. NBodyKit does not seem to make any such guaranty. For small
power spectrum amplitudes that shouldn't matter.


%%%%%%%%%%%%%%%%%%%%%%%%%%%%%%%%%%%%%%%%%%%%%%%%%%%%%
\section{Measuring the Power Spectrum and its Multipoles}

In this section we show how to measure the multipoles of the power spectrum.
The presentation here largely follows \citet{Agrawal+:2017JCAP...10..003A}.


%%%%%%%%%%%%%%%
\subsection{Density Contrast}
Here we will focus on the density contrast
\ba
\delta(\vx) &\equiv \frac{n(\vx)}{\bar n} - 1
\ea
If we assign a galaxy to its nearest grid point, then we estimate the density
contrast as
\ba
\label{eq:delta_x_winfn}
1 + \delta^\obs(\vx) &= \int\dd^3x' \, W(\vx-\vx') \(1 + \delta(\vx')\)
\ea
More on grid assignment in \cref{sec:grid_assignment}
Then,
\ba
\delta^\obs(\vk)
%&= \int\dd^3x \, e^{-i\vk\cdot\vx} \, \delta^\obs(\vx)
= \widetilde W(\vk)\, \delta(\vk)\,.
\ea
For discrete galaxies at positions $\vx_g$ the density is modeled as
$\(1+\delta(\vx')\)\bar{n}=\sum_g\delta^D(\vx'-\vx_g)$. Hence, if we assign
galaxies to their nearest grid point $\vx$, then the observed density contrast
at $\vx$ will be given by \refeq{delta_x_winfn} with a top-hat window function,
\ba
W(\vx-\vx') &= \prod_i \frac{1}{H} \, \mathcal{T}\(\frac{x_i-x'_i}{H}\)\,,
\ea
where the product is over the three dimensions $x$, $y$, and $z$, and the
top-hat is defined as
\ba
\mathcal{T}(x) &= \begin{cases}
    1 & \text{if } |x| < \frac12\,, \\
    \frac12 & \text{if } |x| = \frac12\,, \\
    0 & \text{otherwise.}
\end{cases}
\ea
Then,
\ba
\widetilde W(\vk)
&= \prod_i\int_{-1/2}^{1/2}\dd u_i \, e^{-iHk_iu_i}
\vs
&= \prod_i j_0\!\(\frac{Hk_i}{2}\)
\vs
&= \prod_i j_0\!\(\frac{L_i k_i}{2n_i}\)
\ea

Including the alias effect, we have $\widetilde W(\vk)=1$ for NGP.


%%%%%%%%%%%%%%%%%%%%%%%%%
\subsection{Discrete Fourier Transform}
We definet the Fourier transform as
\ba
\delta(k)
&= \int\dd x\, e^{-ikx} \, \delta(x)
\vs
&= \frac{L}{N} \sum_n e^{-iLkn/N} \, \delta(x_n)
\vs
&= \frac{L}{N} \, \mathrm{FFT}\left[\delta(x)\right](k)
\ea
with $k_F=2\pi/L$.
and the inverse
\ba
\delta(x)
&= \int\frac{\dd k}{2\pi}\,e^{ikx}\,\delta(k)
\vs
&= \frac{1}{L}\,\frac{N}{N}\sum_k e^{ikx}\,\delta(k)
\vs
&= \frac{N}{L}\,\mathrm{IFFT}\left[\delta(k)\right](x)
\ea




%%%%%%%%%%%%%%%%
\subsection{Measuring the full Power Spectrum}
The power spectrum is
\ba
\hat P(k,\mu,\phi)
&= W_\mathrm{mesh,a}^{-1}(\vk) \, W_\mathrm{mesh,b}^{-1}(\vk)
  \vs&\quad\times
\Big[
  \Re\(\delta_a^*(\vk) \delta_b(\vk)\)
  %\vs&\qquad
  - P_\mathrm{shot,ab}
\Big]\,,
\ea
where
\ba
W_\mathrm{mesh}
&= \left[
\mathop{j_0}\!\(\frac{k_xH}{2}\)
\mathop{j_0}\!\(\frac{k_yH}{2}\)
\mathop{j_0}\!\(\frac{k_zH}{2}\)
\right]^p\,,
\ea
where $j_0(x)=\sin(x)/x$ is the spherical Bessel function.
For NGP, $p=1$, for CIC, $p=2$, for TSC, $p=3$.
The shot noise is
\ba
P_\mathrm{shot,ab}
&= \frac{\delta^K_{ab}}{\bar n_a}\,.
\ea


%%%%%%%%%%%%%%%%
\subsection{Measuring Multipoles of the Power Spectrum}
The multipole power spectrum $P_\ell(k)$ is defined s.t.\ 
\ba
\label{eq:multipole_def}
P(k,\mu) &= \sum_{l=0}^\infty P_l(k) \mathcal{L}_l(\mu)\,,
\ea
where $\mathcal{L}_m(\mu)$ are the Legendre polynomials. The Legendre
polynomials obey the orthogonality relation
\ba
\int_{-1}^1\dd\mu\,\mathcal{L}_m(\mu)\,\mathcal{L}_n(\mu)
&= \frac{2}{2m+1}\,\delta^K_{mn}\,.
\ea
This allows us to invert \refeq{multipole_def}. When measuring the power
spectrum from data on a grid, however, we must discretize the integral. Then,
as a first estimate for the multipoles, we obtain
\ba
\label{eq:multipole_brute_estimate}
\hat P_m(k)
&= \sum_{\mu,\phi} \hat P(k,\mu,\phi) \, \mathcal{L}_m(\mu)
%= \sum_{\mu,\phi} \sum_{l=0}^\infty P_l(k)\,\mathcal{L}_l(\mu)\,\mathcal{L}_m(\mu)
= \sum_{l=0}^\infty P_l(k) \, \mathcal{M}_{lm}(k)\,,
\ea
where we assume that the power spectrum is independent of $\phi$ and
\ba
\mathcal{M}_{lm}(k)
&= \sum_{\mu,\phi} \mathcal{L}_l(\mu)\,\mathcal{L}_m(\mu)
\ea
is the $\mu$-leakage matrix. The $k$-dependence comes from the sum over $\mu$,
which depends on the grid of the data. Inverting
\refeq{multipole_brute_estimate}, we get a precise estimate for the multipole
moments
\ba
P_l(k) &= \sum_{m=0}^{l_\max} \hat P_m(k) \, \mathcal{M}^{-1}_{lm}(k)\,.
\ea
The sum over $m$ should go from zero to infinity. However, if there are only
$N_k=\mathcal{M}_{00}$ modes, then we can only measure up to $l=N_k-1$. For
larger multipoles, $\mathcal{M}$ will be singular and not invertible.
Furthermore, we only consider multipoles up to some $l_\mathrm{max}$. Thus, for
a given mode, the $\mu$-leakage matrix $\mathcal{M}_{lm}$ will only consist of
entries $l,m\leq\min(N_k-1,l_\mathrm{max})$.

Due to the finite width of the $k$-bins, we must estimate the mean $k$ at which
the multipole moment is measured. We do this by taking the average $k$ within
the bin, giving each mode the same weight. That is
\ba
k &= \frac{1}{N_k} \sum_{\mu,\phi} k_i\,,
\ea
where the sum goes over all the modes $k_i$ within the $k$-bin, and
$N_k=\mathcal{M}_{00}$ is the number of modes.


%%%%%%%%%%%%%%%%%
\subsection{Grid Assignment}
\label{sec:grid_assignment}
Here we detail the grid assignment schemes NGP, CiC, TSC, PSC. The key question
is how to estimate the number density $n(\vr)$ on a grid from discrete points.

Following \citet{Birdsall+:1969JCoPh...3..494B, Jing:2005ApJ...620..559J}, each
galaxy is modeled as having a shape $S(\vx - \vx_i)$. Whichever portion of that
shape falls within a given grid cell is then the weight assigned to that grid
cell.

The shape function for a galaxy at position $\vx_i$ considered are
\ba
S_\mathrm{NGP}(\vx - \vx_i) &= \delta^D\!\(\frac{\vx - \vx_i}{H}\)\,,
\\
S_\mathrm{CIC}(\vx - \vx_i) &= [\delta^D \star \mathcal{T}^3]\!\(\frac{\vx - \vx_i}{H}\)\,,
\\
S_\mathrm{TSC}(\vx - \vx_i) &= [\delta^D \star \mathcal{T}^3 \star \mathcal{T}^3]\!\(\frac{\vx - \vx_i}{H}\)\,,
\\
S_\mathrm{PSC}(\vx - \vx_i) &= [\delta^D \star \mathcal{T}^3 \star \mathcal{T}^3 \star \mathcal{T}^3]\!\(\frac{\vx - \vx_i}{H}\)\,,
\ea
where $\star$ denotes a convolution. Explicitly,
\ba
S_\mathrm{CIC}(\vx - \vx_i)
&= \int\frac{\dd^3r}{H^3}\, \delta^D\!\(\frac{\vx - \vx_i - \vr}{H}\) \mathcal{T}^3\!\(\frac{\vr}{H}\)
\\
&= \mathcal{T}^3\!\(\frac{\vx - \vx_i}{H}\)\,.
\ea
Or,
\ba
S_\mathrm{TSC}(\vx - \vx_i)
&= \int\frac{\dd^3r}{H^3}\, \mathcal{T}^3(\vx - \vx_i - \vr) \mathcal{T}^3\!\(\frac{\vr}{H}\).
\ea
And
\begin{widetext}
\ba
S_\mathrm{PSC}(\vx - \vx_i)
&= \int\dd^3r'\, S_\mathrm{TSC}(\vx - \vx_i - \vr') \mathcal{T}^3\!\(\vr'\)
\\
&= \int\dd^3r'
\int\dd^3r \mathcal{T}^3(\vx - \vx_i - \vr' - \vr) \mathcal{T}^3(\vr)
\mathcal{T}^3\!\(\vr'\)
\\
&= \int\dd^3r''\, \mathcal{T}^3(\vr'')
\int\dd^3r'\, \mathcal{T}^3(\vr')
\int\dd^3r\, \mathcal{T}^3(\vr)
\,\delta^D(\vx - \vx_i - \vr - \vr' - \vr'')
\\
&= \prod_{\mu\in\{x,y,z\}}
\int\dd r_\mu''\, \mathcal{T}(r_\mu'')
\int\dd r_\mu'\, \mathcal{T}(r_\mu')
\int\dd r_\mu\, \mathcal{T}(r_\mu)
\,\delta^D(x_\mu - x^i_\mu - r_\mu - r_\mu' - r_\mu'')
\\
&= \prod_{\mu\in\{x,y,z\}}
\int_{-\frac12}^\frac12\dd r_\mu''
\int_{-\frac12}^\frac12\dd r_\mu'
\int_{-\frac12}^\frac12\dd r_\mu
\,\delta^D(x_\mu - x^i_\mu - r_\mu - r_\mu' - r_\mu'')
\\
&= \prod_{\mu\in\{x,y,z\}}
\int_{-\frac12}^\frac12\dd r_\mu'
\int_{-\frac12}^\frac12\dd r_\mu
\begin{cases}
  1 & \text{ for } -\frac12 < x_\mu - x^i_\mu - r_\mu - r_\mu' < \frac12\,,
  \\
  0 & \text{ otherwise. }
\end{cases}
\\
&= \prod_{\mu\in\{x,y,z\}}
\int_{-\frac12}^\frac12\dd r_\mu'
\int_{-\frac12}^\frac12\dd r_\mu
\begin{cases}
  1 & \text{ for } -\frac12 < x_\mu - x^i_\mu - r_\mu - r_\mu' < \frac12\,,
  \\
  0 & \text{ otherwise. }
\end{cases}
\ea
\end{widetext}
Nah, there's got to be a better way. The Fourier transform of the top-hat is
\ba
\widetilde{\mathcal{T}}(k_\mu) = \frac{\sin\!\(\frac{H k_\mu}{2}\)}{\frac{H k_\mu}{2}}.
\ea
Since the shape function is a convolution of several top-hats,
\ba
S_p(\vx - \vx_i) = [\star\mathcal{T}]^{p-1}(\vx - \vx_i)\,,
\ea
by the convolution theorem we get in Fourier space
\ba
\widetilde S_p(\vk) &= \widetilde{\mathcal{T}}^{p-1}(\vk)\,.
\ea
And
\ba
S_p(\vx - \vx_i)
&= \int\frac{\dd^3k}{(2\pi)^3}\,e^{i \vk\cdot(\vx - \vx_i)}\,\widetilde S_p(\vk)
\\
&= \int\frac{\dd^3k}{(2\pi)^3}\,e^{i \vk\cdot(\vx - \vx_i)}\,\widetilde{\mathcal{T}}^p(\vk)
\ea
In the $x$-direction, substituting $\kappa=Hk_x/2$,
\ba
S_p(x - x_i)
&= \int\frac{\dd k_x}{2\pi}\,e^{i k_x (x - x_i)}\,\widetilde{\mathcal{T}}^p(k_x)
\\
&= \frac{2}{H}\int\frac{\dd \kappa}{2\pi}\,e^{i \kappa \frac{2}{H}(x - x_i)}
\left[\frac{\sin(\kappa)}{\kappa}\right]^p.
\ea


%%%%%%%%%%%%%%%%%%%%%%%%%%%%%%%
\subsection{Assignment results}
%%%%%%%%%%%%%%%%%%% Realspace %%%%%%%%%%%%%%%%%
\begin{figure*}
  \centering
  \incgraph{../figs/pkest_realspace_nmesh256_simNGP_estNGP_fxshift0.0_pkest_rdiff.pdf}
  \incgraph{../figs/pkest_realspace_nmesh256_simNGP_estNGP_fxshift0.5_pkest_rdiff.pdf}
  \incgraph{../figs/pkest_realspace_nmesh512_simNGP_estNGP_fxshift0.0_pkest_rdiff.pdf}
  \incgraph{../figs/pkest_realspace_nmesh512_simNGP_estNGP_fxshift0.5_pkest_rdiff.pdf}
  \caption{Real space, sim:NGP, est:NGP. Top: Nmesh=256. Bottom: Nmesh=512}
  \label{fig:grid_assignment_realspace}
\end{figure*}
%%%%%%%%%
\begin{figure*}
  \centering
  %\incgraph{../figs/pkest_realspace_simCIC_estNGP_fxshift0.0_pkest.pdf}
  %\incgraph{../figs/pkest_realspace_simCIC_estNGP_fxshift0.5_pkest.pdf}
  \incgraph{../figs/pkest_realspace_nmesh256_simCIC_estNGP_fxshift0.0_pkest_rdiff.pdf}
  \incgraph{../figs/pkest_realspace_nmesh256_simCIC_estNGP_fxshift0.5_pkest_rdiff.pdf}
  \incgraph{../figs/pkest_realspace_nmesh512_simCIC_estNGP_fxshift0.0_pkest_rdiff.pdf}
  \incgraph{../figs/pkest_realspace_nmesh512_simCIC_estNGP_fxshift0.5_pkest_rdiff.pdf}
  \caption{Real space, sim:CIC, est:NGP. Top: Nmesh=256. Bottom: Nmesh=512.}
  \label{fig:grid_assignment_realspace_CIC}
\end{figure*}
%%%%%%%%%
\begin{figure*}
  \centering
  \incgraph{../figs/pkest_realspace_nmesh256_simCIC_estNGP_fxshift0.0_p3_pkest_rdiff.pdf}
  \incgraph{../figs/pkest_realspace_nmesh256_simCIC_estNGP_fxshift0.5_p3_pkest_rdiff.pdf}
  \incgraph{../figs/pkest_realspace_nmesh512_simCIC_estNGP_fxshift0.0_p3_pkest_rdiff.pdf}
  \incgraph{../figs/pkest_realspace_nmesh512_simCIC_estNGP_fxshift0.5_p3_pkest_rdiff.pdf}
  \caption{Real space, sim:CIC, est:NGP. Top: Nmesh=256. Bottom: Nmesh=512, compensated.}
  \label{fig:grid_assignment_realspace_CIC_p3}
\end{figure*}
%%%%%%%%%
\begin{figure*}
  \centering
  %\incgraph{../figs/pkest_realspace_simCIC_estNGP_fxshift0.0_pkest.pdf}
  %\incgraph{../figs/pkest_realspace_simCIC_estNGP_fxshift0.5_pkest.pdf}
  \incgraph{../figs/pkest_realspace_nmesh256_simNGP_estCIC_fxshift0.0_pkest_rdiff.pdf}
  \incgraph{../figs/pkest_realspace_nmesh256_simNGP_estCIC_fxshift0.5_pkest_rdiff.pdf}
  \incgraph{../figs/pkest_realspace_nmesh256_simCIC_estCIC_fxshift0.0_pkest_rdiff.pdf}
  \incgraph{../figs/pkest_realspace_nmesh256_simCIC_estCIC_fxshift0.5_pkest_rdiff.pdf}
  \caption{Real space, Top: sim:NGP est:CIC. Bottom: sim:CIC est:CIC. Nmesh=256.}
  \label{fig:grid_assignment_realspace_CIC_NGP}
\end{figure*}
%%%%%%%%%
%%%%%%%%%%%%%%%%%%% Redshift space %%%%%%%%%%%%%%%%%
\begin{figure*}
  \centering
  \incgraph{../figs/pkest_redshiftspace_nmesh256_simCIC_estNGP_velo1_fxshift0.0_pkest_rdiff.pdf}
  \incgraph{../figs/pkest_redshiftspace_nmesh256_simCIC_estNGP_velo1_fxshift0.5_pkest_rdiff.pdf}
  \caption{Redshift space, sim:CIC, est:NGP, velocity 1.}
  \label{fig:grid_assignment_redshiftspace_CIC_velo1}
\end{figure*}
%%%%%%%%%
\begin{figure*}
  \centering
  \incgraph{../figs/pkest_redshiftspace_nmesh256_simCIC_estNGP_velo2_fxshift0.0_pkest_rdiff.pdf}
  \incgraph{../figs/pkest_redshiftspace_nmesh256_simCIC_estNGP_velo2_fxshift0.5_pkest_rdiff.pdf}
  \caption{Redshift space, sim:CIC, est:NGP, velocity 2.}
  \label{fig:grid_assignment_redshiftspace_CIC_velo2}
\end{figure*}
%%%%%%%%%
\begin{figure*}
  \centering
  \incgraph{../figs/pkest_redshiftspace_nmesh256_simCIC_estNGP_velo3_fxshift0.0_pkest_rdiff.pdf}
  \incgraph{../figs/pkest_redshiftspace_nmesh256_simCIC_estNGP_velo3_fxshift0.5_pkest_rdiff.pdf}
  \caption{Redshift space, sim:CIC, est:NGP, velocity 3.}
  \label{fig:grid_assignment_redshiftspace_CIC_velo3}
\end{figure*}
%%%%%%%%%
\begin{figure*}
  \centering
  \incgraph{../figs/pkest_redshiftspace_nmesh256_simCIC_estNGP_velo4_fxshift0.0_pkest_rdiff.pdf}
  \incgraph{../figs/pkest_redshiftspace_nmesh256_simCIC_estNGP_velo4_fxshift0.5_pkest_rdiff.pdf}
  \caption{Redshift space, sim:CIC, est:NGP, velocity 4.}
  \label{fig:grid_assignment_redshiftspace_CIC_velo4}
\end{figure*}
%%%%%%%%%
\begin{figure*}
  \centering
  \incgraph{../figs/pkest_redshiftspace_nmesh256_simCIC_estNGP_velo5_fxshift0.0_pkest_rdiff.pdf}
  \incgraph{../figs/pkest_redshiftspace_nmesh256_simCIC_estNGP_velo5_fxshift0.5_pkest_rdiff.pdf}
  \incgraph{../figs/pkest_redshiftspace_nmesh512_simCIC_estNGP_velo5_fxshift0.0_pkest_rdiff.pdf}
  \incgraph{../figs/pkest_redshiftspace_nmesh512_simCIC_estNGP_velo5_fxshift0.5_pkest_rdiff.pdf}
  \caption{Redshift space, sim:CIC, est:NGP, velocity 5.}
  \label{fig:grid_assignment_redshiftspace_CIC_velo5}
\end{figure*}
%%%%%%%%%
\begin{figure*}
  \centering
  \incgraph{../figs/pkest_redshiftspace_nmesh256_simCIC_estNGP_velo5_fxshift0.0_p3_pkest_rdiff.pdf}
  \incgraph{../figs/pkest_redshiftspace_nmesh256_simCIC_estNGP_velo5_fxshift0.5_p3_pkest_rdiff.pdf}
  \incgraph{../figs/pkest_redshiftspace_nmesh512_simCIC_estNGP_velo5_fxshift0.0_p3_pkest_rdiff.pdf}
  \incgraph{../figs/pkest_redshiftspace_nmesh512_simCIC_estNGP_velo5_fxshift0.5_p3_pkest_rdiff.pdf}
  \caption{Redshift space, sim:CIC, est:NGP, velocity 5, corrected.}
  \label{fig:grid_assignment_redshiftspace_CIC_velo5_p3}
\end{figure*}
%%%%%%%%%
\begin{figure*}
  \centering
  \incgraph{../figs/pkest_redshiftspace_nmesh256_simCIC_estNGP_velo6_fxshift0.0_pkest_rdiff.pdf}
  \incgraph{../figs/pkest_redshiftspace_nmesh256_simCIC_estNGP_velo6_fxshift0.5_pkest_rdiff.pdf}
  \incgraph{../figs/pkest_redshiftspace_nmesh512_simCIC_estNGP_velo6_fxshift0.0_pkest_rdiff.pdf}
  \incgraph{../figs/pkest_redshiftspace_nmesh512_simCIC_estNGP_velo6_fxshift0.5_pkest_rdiff.pdf}
  \caption{Redshift space, sim:CIC, est:NGP, velocity 6.}
  \label{fig:grid_assignment_redshiftspace_CIC_velo6}
\end{figure*}
%%%%%%%%%
\begin{figure*}
  \centering
  \incgraph{../figs/pkest_redshiftspace_nmesh256_simCIC_estNGP_velo6_fxshift0.0_p3_pkest_rdiff.pdf}
  \incgraph{../figs/pkest_redshiftspace_nmesh256_simCIC_estNGP_velo6_fxshift0.5_p3_pkest_rdiff.pdf}
  \incgraph{../figs/pkest_redshiftspace_nmesh512_simCIC_estNGP_velo6_fxshift0.0_p3_pkest_rdiff.pdf}
  \incgraph{../figs/pkest_redshiftspace_nmesh512_simCIC_estNGP_velo6_fxshift0.5_p3_pkest_rdiff.pdf}
  \caption{Redshift space, sim:CIC, est:NGP, velocity 6, corrected.}
  \label{fig:grid_assignment_redshiftspace_CIC_velo6_p3}
\end{figure*}
%%%%%%%%%



%%%%%%%%%%%%%%%%%%%%%%%%%%%%%%%
%\subsection{Multipole Error Estimate}







\end{document}



% vim: set sw=2 sts=2 et :
